%%
%% This is file `sample-sigconf.tex',
%% generated with the docstrip utility.
%%
%% The original source files were:
%%
%% samples.dtx  (with options: `sigconf')
%% 
%% IMPORTANT NOTICE:
%% 
%% For the copyright see the source file.
%% 
%% Any modified versions of this file must be renamed
%% with new filenames distinct from sample-sigconf.tex.
%% 
%% For distribution of the original source see the terms
%% for copying and modification in the file samples.dtx.
%% 
%% This generated file may be distributed as long as the
%% original source files, as listed above, are part of the
%% same distribution. (The sources need not necessarily be
%% in the same archive or directory.)
%%
%% The first command in your LaTeX source must be the \documentclass command.
\documentclass[manuscript,screen]{acmart}
%% NOTE that a single column version may be required for 
%% submission and peer review. This can be done by changing
%% the \doucmentclass[...]{acmart} in this template to 
%% \documentclass[manuscript,screen]{acmart}
%% 
%% To ensure 100% compatibility, please check the white list of
%% approved LaTeX packages to be used with the Master Article Template at
%% https://www.acm.org/publications/taps/whitelist-of-latex-packages 
%% before creating your document. The white list page provides 
%% information on how to submit additional LaTeX packages for 
%% review and adoption.
%% Fonts used in the template cannot be substituted; margin 
%% adjustments are not allowed.
%%
%%
%% \BibTeX command to typeset BibTeX logo in the docs
\AtBeginDocument{%
  \providecommand\BibTeX{{%
    \normalfont B\kern-0.5em{\scshape i\kern-0.25em b}\kern-0.8em\TeX}}}

%% CUSTOM PACKAGES
\usepackage{multirow}
\usepackage{subcaption}

%% CUSTOM CODE
\usepackage[compact]{titlesec}         % you need this package
\titlespacing{\section}{0pt}{0pt}{0pt} % this reduces space between (sub)sections to 0pt, for example
\AtBeginDocument{%                     % this will reduce spaces between parts (above and below) of texts within a (sub)section to 0pt, for example - like between an 'eqnarray' and text
  \setlength\abovedisplayskip{0pt}
  \setlength\belowdisplayskip{0pt}}

\usepackage[rawfloats=true]{floatrow} 
\restylefloat{figure}     % this, I think, will reduce spaces between images and text.  

% \vspace{-3mm}

%% Rights management information.  This information is sent to you
%% when you complete the rights form.  These commands have SAMPLE
%% values in them; it is your responsibility as an author to replace
%% the commands and values with those provided to you when you
%% complete the rights form.
\setcopyright{acmcopyright}
\copyrightyear{2022}
\acmYear{2022}
\acmDOI{XXXXXXX.XXXXXXX}

%% These commands are for a PROCEEDINGS abstract or paper.
\acmConference[ICCMS '22]{}{June 24--26, 2022}{Chongqing, China}
%
%  Uncomment \acmBooktitle if th title of the proceedings is different
%  from ``Proceedings of ...''!
%
\acmBooktitle{Chongqing '22: 4th International Conference on Computer Modeling and Simulation, June 24--26, 2022, Chongqing, China} 
% \acmPrice{15.00}
% \acmISBN{978-1-4503-XXXX-X/22/06}

%%
%% Submission ID.
%% Use this when submitting an article to a sponsored event. You'll
%% receive a unique submission ID from the organizers
%% of the event, and this ID should be used as the parameter to this command.
%%\acmSubmissionID{123-A56-BU3}

%%
%% The majority of ACM publications use numbered citations and
%% references.  The command \citestyle{authoryear} switches to the
%% "author year" style.
%%
%% If you are preparing content for an event
%% sponsored by ACM SIGGRAPH, you must use the "author year" style of
%% citations and references.
%% Uncommenting
%% the next command will enable that style.
%%\citestyle{acmauthoryear}

%%
%% end of the preamble, start of the body of the document source.
\begin{document}

%%
%% The "title" command has an optional parameter,
%% allowing the author to define a "short title" to be used in page headers.
\title{Dynamic models for analysing stock market behaviour under the COVID-19 pandemic}

%%
%% The "author" command and its associated commands are used to define
%% the authors and their affiliations.
%% Of note is the shared affiliation of the first two authors, and the
%% "authornote" and "authornotemark" commands
%% used to denote shared contribution to the research.
\author{Maarten Peters}
% \authornote{Main author}
\email{maarten.peters@gmail.com}
\affiliation{%
  \institution{University of Amsterdam}
  \streetaddress{Science Park 904}
  \city{Amsterdam}
  % \state{Ohio}
  \country{the Netherlands}
  \postcode{1098 XH}
}

\author{Dr. Cristian Rodriguez Rivero}
\authornote{Internal Supervisor from the University of Amsterdam as part of the master thesis programme}
\email{c.m.rodriguezrivero@uva.nl}
\affiliation{%
  \institution{University of Amsterdam, FNWI, IvI}
  \streetaddress{Science Park 904}
  \city{Amsterdam}
  % \state{Ohio}
  \country{the Netherlands}
  \postcode{1098 XH}
}

\author{Dr. Julián Antonio Pucheta}
\authornote{External Supervisor from the National University of Córdoba}
\email{jpucheta@unc.edu.ar}
\affiliation{%
  \institution{Universidad Nacional de Córdoba}
  \streetaddress{Av. Haya de la Torre}
  \city{Córdoba}
  % \state{Ohio}
  \country{Argentina}
  \postcode{HR77+RQ}
}

%%
%% By default, the full list of authors will be used in the page
%% headers. Often, this list is too long, and will overlap
%% other information printed in the page headers. This command allows
%% the author to define a more concise list
%% of authors' names for this purpose.
\renewcommand{\shortauthors}{Maarten Peters, et al.}

%%
%% The abstract is a short summary of the work to be presented in the
%% article.
\begin{abstract}
During the COVID-19 pandemic \cite{velavan2020covid} questions were raised on how to balance government measures ensuring population health and allowing economic development. At the start of the pandemic, Saudi-Arabia and Russia were in a pricing war over crude oil \cite{jacobs2020opec} which, along with speculation on the economic impact of COVID-19, led to a unprecedented negative crude oil price in the West Texas Intermediate (WTI) \cite{CORBET2020104978}. As the WTI serves as a benchmark for crude oil prices in North America, and a proxy for economic development \cite{kaufmann2011role}, it is an interesting candidate to use for price forecasting \cite{YU20082623}. The pandemic provides a unique perspective, as it introduces a new set of variables \cite{dong2020interactive, hale2020variation}, such as infections, deaths, vaccinations and government measures \footnote{Provided by the National Institute for Public Health and the Environment (in Dutch: Rijksinstituut voor Volksgezondheid en Milieu, RIVM), the Johns Hopkins University Coronavirus Resource Center (JHU) and the Oxford COVID-19 Government Response Tracker (OxCGRT).}, that might aid in predicting economic development \cite{chudik2020economic, baldwin2020economics, fernandes2020economic}. Related studies generally focus on macroeconomic development, such as gross domestic product (GDP), unemployment or inflation over years or decades, rather than short-term development over days, weeks or months. This study attempts to combine data from the COVID-19 pandemic, weather, stock pricing data and machine learning techniques to determine the relationship between these variables and their value towards more accurate price forecasting. As stock prices have high variance, extreme values might indicate local or global stock market crashes, an optimal model would be able to predict these crashes. To determine the outcome of our research question, we compare the value of our data between a baseline model, linear model and two state-of-the-art (SOTA) models, the random forest regressor (RFR) and auto-regressive integrated moving average (ARIMA) model. Both SOTA models tend to perform better or similar with less features, indicating the data does not add significant value to the prediction of stock market values.\footnote{An overview of the code, documentation and data collected and/or produced during this research has been made available on Github: \url{https://github.com/maartenpeters/dynamic-models-for-stock-market-behaviour.git}}
\end{abstract}

%%
%% The code below is generated by the tool at http://dl.acm.org/ccs.cfm.
%% Please copy and paste the code instead of the example below.
%%
%%\begin{CCSXML}
%%<ccs2012>
 %%<concept>
%%  <concept_id>10010520.10010553.10010562</concept_id>
%%  <concept_desc>Computer systems organization~Embedded systems</concept_desc>
%%  <concept_significance>500</concept_significance>
 %%</concept>
 %%<concept>
%%  <concept_id>10010520.10010575.10010755</concept_id>
%%  <concept_desc>Computer systems organization~Redundancy</concept_desc>
%%  <concept_significance>300</concept_significance>
 %%</concept>
 %%<concept>
%%  <concept_id>10010520.10010553.10010554</concept_id>
%%  <concept_desc>Computer systems organization~Robotics</concept_desc>
%%  <concept_significance>100</concept_significance>
 %%</concept>
 %%<concept>
%%  <concept_id>10003033.10003083.10003095</concept_id>
%%  <concept_desc>Networks~Network reliability</concept_desc>
%%  <concept_significance>100</concept_significance>
 %%</concept>
%%</ccs2012>
%%\end{CCSXML}
%%
%%\ccsdesc[500]{Computer systems organization~Embedded systems}
%%\ccsdesc[300]{Computer systems organization~Redundancy}
%%\ccsdesc{Computer systems organization~Robotics}
%%\ccsdesc[100]{Networks~Network reliability}

\begin{CCSXML}
<ccs2012>
   <concept>
       <concept_id>10010147.10010257.10010293.10010307</concept_id>
       <concept_desc>Computing methodologies~Learning linear models</concept_desc>
       <concept_significance>300</concept_significance>
       </concept>
 </ccs2012>
\end{CCSXML}

\ccsdesc[300]{Computing methodologies~Learning linear models}

%%
%% Keywords. The author(s) should pick words that accurately describe
%% the work being presented. Separate the keywords with commas.
\keywords{COVID-19, dynamic processes, stock market index, multivariable analysis, time series forecasting}

%% A "teaser" image appears between the author and affiliation
%% information and the body of the document, and typically spans the
%% page.
% \begin{teaserfigure}
%   \includegraphics[width=\textwidth]{sampleteaser}
%   \caption{Seattle Mariners at Spring Training, 2010.}
%   \Description{Enjoying the baseball game from the third-base
%   seats. Ichiro Suzuki preparing to bat.}
%   \label{fig:teaser}
% \end{teaserfigure}

%%
%% This command processes the author and affiliation and title
%% information and builds the first part of the formatted document.
\maketitle

\section{Introduction}
\label{sec:intro}
At the start and during the COVID-19 pandemic, rising infection rates, deaths and widespread lockdown measures raised questions on how to limit the pandemic's impact on economic development \cite{velavan2020covid}. This thesis looks into how the pandemic affects stock market values, as these are notoriously volatile in their pricing and influenced by a wide range of factors, demonstrated by the WTI value dipping in the negative at the first wave of infections \cite{CORBET2020104978}. 
The relationship between stock markets and economic factors falls under the study of macroeconomics and finds it origins in the business cycle theory, dating back to the early 1800s \cite{de1827nouveaux}. This theory generally states that all economic development follows a cycle or waveform, going through the phases expansion, crisis, recession and recovery. These business cycles can be measured \cite{baxter1999measuring} and are part of the tools that a government can use to mitigate economic downturn. Predicting the development of a business cycle is generally based on economic indicators and economic theory, such as Keynesian theory  \cite{keynes2018general} and real-business cycle (RBC) theory \cite{kydland1982time}. One of the generally accepted economic indicators is the Standard and Poor's (S\&P) 500 as an indicator for the United States (US) economy, stating a relationship between economic development and the top 500 US companies. % \cite{diblasi_2021}
One of the complexities of these relationships, between a country and a national representation of the local stock market, is that a stock market is not a local phenomenon. Stocks and other financial products are traded internationally via a stock exchange and these stock indexes, such as the S\&P500, are a representation of the stock exchange itself and not (necessarily) of the country it resides in. As the pandemic's impact and government restrictions varies between countries, it is needed to isolate a region to perform a causal analysis of its impact on a stock market.
When the WTI value dropped, all oil-related stocks dropped in value as a consequence, such as Royal Dutch Shell, one of the largest companies in the Netherlands. As previous research noted a relationship between pandemic data and economic development, and a relationship between economic development and stock indexes, the question can be posed whether these relationships aid in predictive modelling of stock values in a region. This research will focus on the Netherlands, with the Amsterdam Exchange index (AEX) as an analog for the S\&P500, the former featuring the top 25 most traded securities on the Amsterdam Exchange. As macroeconomics looks at the large scale development of an economy, the models generally apply to years or decades of data. Daily stock market development is a lot more volatile than other economic indicators, generally displaying a non-linear patterns. 
Machine learning can aid in these predictions, as it can capture non-linear behaviour better than traditional linear models \cite{zhang2009stock}. Still, stock market prediction is non-trivial as the efficient market hypothesis (EMH) \cite{peters1996chaos} states that no investor can gain an advantage over others based on historical and current information. Some investors base their decisions on fundamental analysis \cite{lev1993fundamental} to gain an advantage on others. Machine learning has shown to be effective at improving prediction accuracy on stock market securities or crude oil pricing \cite{jain1996, YU20082623, zhang2009stock, thawornwong2004forecasting}, especially using recurrent- (RNN) or artificial neural networks (ANN). % \cite{crotty2009} \cite{garber1989}
Having established the relationship between stock market values and economic development, the question remains whether the pandemic actually influenced stock market values. Social and economic factors have shown historical importance in stock market crashes, such as with the Tulip Mania, the Wall Street Crash of 1929 and the Great Depression \cite{white1990, kindleberger1986world} and the global financial crisis of 2008. An important topic during the recent pandemic was whether it would influence the economy and what its impact would be. Historically pandemics have shown their effects \cite{osterholm2017preparing, correia1918pandemics, jorda2020longer}, but the quantitative relationship between the COVID-19 pandemic and economic development is still an object of research \cite{chudik2020economic, baldwin2020economics, fernandes2020economic, deb2020economic}. During the pandemic daily data was registered on its development, such as infection rates, deaths, vaccinations and government restrictions \cite{dong2020interactive, hale2020variation}. As stock market crashes generally are a social phenomenon, this research investigates the relationship between these features and stock market pricing. The following research questions are posed:
\begin{itemize}
    \item[RQ1] \ldots To what extent does the COVID-19 pandemic data influence stock market values?
        \begin{itemize}
            \item What is the baseline performance for standard statistical models on short term and long term?
            \item How do different statistical and machine learning models compare given the same data?
            \item How do models generalize to different time-periods, such as distinctions between different infection waves?
            \item Which variables in a pandemic affect stock market pricing and to what extent? % Variables such as ...
            % \begin{itemize}
            %     \item COVID-19 pandemic data incl.: infections, deaths, vaccinations and % government restrictions
            %     \item Environmental factors such as calendar data or weather
            % \end{itemize}
        \end{itemize}
    \item[RQ2] \ldots To what extent can machine learning techniques aid in predicting stock market pricing, given a comparison of baseline models and the addition of pandemic data?
\end{itemize}
As the pandemic influenced a wide array of human behaviours, variables are restricted to the ones mentioned above. Other variables, such as mobility, energy consumption and social contacts have shown large differences during the pandemic, but are outside of the scope of this research.
% \paragraph{Overview of paper}
% To create context and support of the hypothesis, this paper first discusses related work \ref{sec:rel}. With this context, a proper methodology \ref{sec:meth} is formulated with an explanation on the data collection \ref{sec:data}, model development \ref{sec:models} and how these answer the proposed research questions \ref{sec:eval}. Using the methodology and proper context from related work, results can be analysed \ref{sec:res}, discussed \ref{sec:dis} and be made to construct a conclusion \ref{sec:conc}.
\input{thesis/2_related_work}
\input{thesis/3_methodology}
\input{thesis/4_evaluation}
\section{Discussion}
\label{sec:dis}
When viewing the results compared to the baseline model, better performance was to be expected from the ARIMA and RFR models. These models should have been able to capture the variance of the non-linear time-series, when properly tuned and fitted. The additional features presented by the COVID-19 pandemic do seem to sporadically add accuracy, however the significance tests show that these are most likely a random or coincidental occurrence rather than a relationship in the data. Linear regression actually consistently performed worse than the baseline, which indicates that the data is non-linear.
In light of related work, this could be due to the methodology of automated model fitting and tuning. Previous studies of forecasting models generally approach these tasks as an iterative process, where a single model is developed on very specific data with large amounts of manual tuning. In these results large parts of the test scores already needed to be discarded due to obvious fitting errors caused by noise in the data, preprocessing or failures of the code. Where EDA showed there was potential in the data and models, when implemented on a larger scale it shows that without manual intervention, these models do not generalize. This was somewhat expected, as other related work already showed that stock market prediction relies on domain knowledge or technical analysis. % The EMH states that with asset prices reflecting all available information, there is no possible way of outperforming the market as asset prices only react to new information. 
As the COVID-19 pandemic was new information at the end of 2019, there was a consequential dip in oil futures and related assets, which set the precedent of the pandemic becoming old information. Also it was clear that this dip in oil futures was a result of compounding events rather than just an effect of the pandemic. Another possible explanation for the insignificant results is that there is no causal relationship between our features. Although publicly available data could've influenced the behavioural psychology of traders, this is a relationship primarily determined by the traders' psychology and only has a secondary relation to this study's data. As businesses moved to working remotely from their employees homes, lockdown effects were overcome for most industries and the effect on the economy was consequently negated. Along with quick developments in vaccines, this could've led traders to the conclusion that the pandemic was of little importance after the first wave of infections. These developments were mostly covered in news stories, which are difficult to quantify, but are generally the main source of information for traders. Any relationship with pandemic information is probably only visible on long term analysis, where events on the stock market accumulate over time and variance decreases.
Although technology has come a long way, and SOTA machine learning models have shown promise in varying fields of expertise, it is clear that these models do not learn 'automatically'. A lot of expertise is needed to properly fit and tune a model to very specific data and only then can it be considered as a viable solution for the specific problem in mind. Generalization of models is acquired by training with large amounts of data, but only when the model in question can do the same on very specific data. For instance, models that perform character recognition in images can generalize to other tasks, such as object classification, however this comes with complex engineering of the models and large amounts of prepared data. In this case, the EDA showed promise on prediction models combined with pandemic data. And with large amounts of fitting, tuning and engineering of a single model it might still be possible to achieve a significant result. Still, generalization of such models comes with large volumes of data and preparation, and only when a precedent is set research can work towards a more general solution. In this research experiments were performed with different models, such as Xgboost, but from sample fittings it was clear it would not outperform RFR without additional data or proper manual tuning.
When reviewing our selection of control, or baseline model and other prediction models, the research also admits that there might have been more appropriate models available at the time of writing. If this study would be revisited, one would do right to reconsider the model selection along with possible different error metrics.

\section{Conclusions}
\label{sec:conc}
Summarizing the results and discussion, this study can not confirm there is any added benefit of including COVID-19 pandemic data as a explanatory variable for the purpose of stock market predictions. Considering the individual contributions of environmental factors, such as weather, this study can not establish a significant relationship. The same is applicable for any of the individual COVID-19 pandemic features, where it is concluded that any relation is either coincidental or inconclusive as no significant performance difference can be measured.
This does not mean there is no added benefit of applying SOTA machine learning techniques or models to improve prediction accuracy, as indicated by related work. This study could not establish a significant relationship between pandemic data and stock market values with ARIMA or RFR models, especially compared to the performance of the baseline model. Forming a basis for more complex models, such as LSTM/RNN, is therefore unsustainable as this study can not provide the data these models require to achieve significant results. There remains a possibility for pre-trained models or transfer learning to improve results, however this would require further research to verify. Financial theories debate on whether an algorithm can consistently beat (the majority of) the stock market and as a consequence one can imagine these models are not within the grasp of the general public. Within the scope of this research it is concluded that it is not possible to achieve such a result and leave this as a springboard for further research.
Although this study did not establish a statistical significant result, there is anecdotal evidence of an impact during the first infection wave of the COVID-19 pandemic. With purpose built models and data there is a possibility of constructing accurate time-series forecasts, but this comes at the cost of generalization.

%%
%% The acknowledgments section is defined using the "acks" environment
%% (and NOT an unnumbered section). This ensures the proper
%% identification of the section in the article metadata, and the
%% consistent spelling of the heading.
% \begin{acks}
% The researcher(s) would like to thank the external and internal supervisors Dr. Julian Antonio Pucheta and Dr. Christian Rodriguez Rivero for supporting this research. As the COVID-19 pandemic is still ongoing as of writing (December 2021), it is a trying time for everyone to keep working. And in that light thanks are given to the University of Amsterdam for their patience and support. 
% Finally I, Maarten Peters, would personally like to acknowledge my partner Melissa van der Werf for supporting me during my two year part-time masters at the University of Amsterdam. As my advisors at the university told, a part-time study next to a full-time employment is no trivial matter, but I can recommend it to anybody who is willing to invest their time in developing their own knowledge and skills, as the effort is well worth the result.
% \end{acks}

%%
%% The next two lines define the bibliography style to be used, and
%% the bibliography file.
\bibliographystyle{ACM-Reference-Format}
\bibliography{thesis/0_literature.bib}


\end{document}
\endinput