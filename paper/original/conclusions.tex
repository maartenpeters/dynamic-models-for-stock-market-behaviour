\section{Discussion}
\label{sec:dis}
When viewing the results compared to the baseline model, better performance was to be expected from the ARIMA and RFR models. These models should have been able to capture the variance of the non-linear time-series, when properly tuned and fitted. The additional features presented by the COVID-19 pandemic do seem to sporadically add accuracy, however the significance tests show that these are most likely a random or coincidental occurrence rather than a relationship in the data. Linear regression actually consistently performed worse than the baseline, which indicates that the data is non-linear.

In light of related work, this could be due to the methodology of automated model fitting and tuning. Previous studies of forecasting models generally approach these tasks as an iterative process, where a single model is developed on very specific data with large amounts of manual tuning. In these results large parts of the test scores already needed to be discarded due to obvious fitting errors caused by noise in the data, preprocessing or failures of the code. Where EDA showed there was potential in the data and models, when implemented on a larger scale it shows that without manual intervention, these models do not generalize. This was somewhat expected, as other related work already showed that stock market prediction relies on domain knowledge or technical analysis. The efficient market hypothesis (EMH) states that with asset prices reflecting all available information, there is no possible way of outperforming the market as asset prices only react to new information. 

As the COVID-19 pandemic was new information at the end of 2019, there was a consequential dip in oil futures and related assets, which set the precedent of the pandemic becoming old information. Also it was clear that this dip in oil futures was a result of compounding events rather than just an effect of the pandemic. Another possible explanation for the insignificant results is that there is no causal relationship between our features. Although publicly available data could've influenced the behavioural psychology of traders, this is a relationship primarily determined by the traders' psychology and only has a secondary relation to this study's data. As businesses moved to working remotely from their employees homes, lockdown effects were overcome for most industries and the effect on the economy was consequently negated. Along with quick developments in vaccines, this could've led traders to the conclusion that the pandemic was of little importance after the first wave of infections. These developments were mostly covered in news stories, which are difficult to quantify, but are generally the main source of information for traders. Any relationship with pandemic information is probably only visible on long term analysis, where events on the stock market accumulate over time and variance decreases.

Although technology has come a long way, and SOTA machine learning models have shown promise in varying fields of expertise, it is clear that these models do not learn 'automatically'. A lot of expertise is needed to properly fit and tune a model to very specific data and only then can it be considered as a viable solution for the specific problem in mind. Generalization of models is acquired by training with large amounts of data, but only when the model in question can do the same on very specific data. For instance, models that perform character recognition in images can generalize to other tasks, such as object classification, however this comes with complex engineering of the models and large amounts of prepared data. In this case, the EDA showed promise on prediction models combined with pandemic data. And with large amounts of fitting, tuning and engineering of a single model it might still be possible to achieve a significant result. Still, generalization of such models comes with large volumes of data and preparation, and only when a precedent is set research can work towards a more general solution. In this research experiments were performed with different models, such as Xgboost \cite{chen2016xgboost}, but from sample fittings it was clear it would not outperform RFR without additional data or proper manual tuning.

%  To what extent does the COVID-19 pandemic data influence stock market values?
%  What is the baseline performance for standard statistical models on short term and % long term?
%  How do different statistical and machine learning models compare given the same data?
%  How do models generalize to different time-periods, such as distinctions between % different infection waves?
%  Which variables in a pandemic affect stock market pricing and to what extent? % Variables such as ...
%  COVID-19 pandemic data incl. 
%    Infections (provided by the RIVM and JHU)
%    Deaths (provided by the RIVM and JHU)
%    Vaccinations (provided by the JHU)
%    Government restrictions, such as lockdown measures (provided by the OxCGRT)
%  Environmental factors such as calendar data or weather
%  To what extent can machine learning techniques aid in predicting stock market pricing, given a comparison of baseline models and the addition of pandemic data?
% \end{itemize}

\section{Conclusions}
\label{sec:conc}
Summarizing the results and discussion, this study can not confirm there is any added benefit of including COVID-19 pandemic data as a explanatory variable for the purpose of stock market predictions. Considering the individual contributions of environmental factors, such as weather, this study can not establish a significant relationship. The same is applicable for any of the individual COVID-19 pandemic features, where it is concluded that any relation is either coincidental or inconclusive as no significant performance difference can be measured.
This does not mean there is no added benefit of applying SOTA machine learning techniques or models to improve prediction accuracy, as indicated by related work. This study could not establish a significant relationship between pandemic data and stock market values with ARIMA or RFR models, especially compared to the performance of the baseline model. Forming a basis for more complex models, such as LSTM/RNN, is therefore unsustainable as this study can not provide the data these models require to achieve significant results. There remains a possibility for pre-trained models or transfer learning to improve results, however this would require further research to verify. Financial theories debate on whether an algorithm can consistently beat (the majority of) the stock market and as a consequence one can imagine these models are not within the grasp of the general public. Within the scope of this research it is concluded that it is not possible to achieve such a result and leave this as a springboard for further research.

\subsection{Acknowledgements}
The researcher(s) would like to thank the external and internal supervisors Dr. Julian Antonio Pucheta and Dr. Christian Rodriguez Rivero for supporting this research. As the COVID-19 pandemic is still ongoing as of writing (December 2021), it is a trying time for everyone to keep working. And in that light thanks are given to the University of Amsterdam for their patience and support. 
Finally I, Maarten Peters, would personally like to acknowledge my partner Melissa van der Werf for supporting me during my two year part-time masters at the University of Amsterdam. As my advisors at the university told, a part-time study next to a full-time employment is no trivial matter, but I can recommend it to anybody who is willing to invest their time in developing their own knowledge and skills, as the effort is well worth the result.