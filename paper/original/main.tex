%\documentclass[a4paper,pdf]{article} % gebruik acm style voor je scriptie: [format=acmsmall, screen=true, review=false]{acmart} 
%### ACM template on overleaf

%* You can find the ACM template also on overleaf. You can then use the points below for the content.
%* <https://www.overleaf.com/latex/templates/acm-conference-proceedings-master-template/pnrfvrrdbfw>


\documentclass[sigconf]{acmart} 
% \settopmatter{printacmref=false} % Removes citation information below abstract
\renewcommand\footnotetextcopyrightpermission[1]{} % removes footnote with conference information in first column
% \pagestyle{plain} % removes running headers
%\documentclass[sigconf,format=acmsmall, screen=true, review=false]{acmart} 
\usepackage{amsmath}
\usepackage{amsfonts}
\usepackage{amssymb}
\usepackage{hyperref}
\usepackage{pdfpages} % http://mirror.unl.edu/ctan/macros/latex/contrib/pdfpages/pdfpages.pdf
\usepackage{booktabs} 
% \usepackage[hyphens]{url}
\usepackage{multirow}

\usepackage[utf8]{inputenc}
\usepackage{graphicx}
\usepackage[colorinlistoftodos]{todonotes} % handig voor commentaar: gebruik \todo{}, zie ftp://ftp.fu-berlin.de/tex/CTAN/macros/latex/contrib/todonotes/todonotes.pdf
\usepackage{listings}
 
\usepackage{tcolorbox}
\usepackage{float}
\usepackage{caption}
\usepackage{subcaption}
\usepackage[showframe]{geometry}% http://ctan.org/pkg/geometry

% Bibliography package
\usepackage[
backend=biber,
style=alphabetic,
sorting=ynt
]{biblatex}
\addbibresource{literature.bib}


% when writing in Dutch
%\usepackage[dutch]{babel}
%\selectlanguage{dutch}


% linenumbering  See https://texblog.org/2012/02/08/adding-line-numbers-to-documents/
% \usepackage{lineno}
% \linenumbers

% Keywords command
\providecommand{\keywords}[1]
{
  \small	
  \textbf{\textit{Keywords---}} #1
}

\newcommand{\shorttitle}{Dynamic models for stock market behaviour} % Put your short title here
\begin{document}

%\input{titlepage}
\input{TitlePages/Thesis-Title-Page-DS}  % or use another template

\pagebreak

%\todototoc
%\listoftodos
%\tableofcontents

\pagebreak

\twocolumn

\begin{abstract}
% [CHANGE] 
% Thesis Design
During the COVID-19 pandemic \cite{velavan2020covid} questions were raised on how to balance government measures ensuring population health and allowing economic development. At the start of the pandemic, Saudi-Arabia and Russia were in a pricing war over crude oil \cite{jacobs2020opec} which, along with speculation on the economic impact of COVID-19, led to a unprecedented negative crude oil price in the West Texas Intermediate (WTI) \cite{CORBET2020104978}. As the WTI serves as a benchmark for crude oil prices in North America, and a proxy for economic development \cite{kaufmann2011role}, it is an interesting candidate to use for price forecasting \cite{YU20082623}. The pandemic provides a unique perspective, as it introduces a new set of variables \cite{dong2020interactive, hale2020variation}, such as COVID-19 infections, deaths, vaccinations and government measures \footnote{Provided by the National Institute for Public Health and the Environment (in Dutch: Rijksinstituut voor Volksgezondheid en Milieu, RIVM), the Johns Hopkins University Coronavirus Resource Center (JHU) and the Oxford COVID-19 Government Response Tracker (OxCGRT).}, that might aid in predicting economic development \cite{chudik2020economic, baldwin2020economics, fernandes2020economic}. Related studies generally focus on macroeconomic development, such as gross domestic product (GDP), unemployment or inflation over years or decades, rather than short-term development over days, weeks or months. This study attempts to combine data from the COVID-19 pandemic, weather, stock pricing data and machine learning techniques to determine the relationship between these variables and their value towards more accurate price forecasting. As stock prices have high variance, extreme values might indicate local or global stock market crashes, an optimal model would be able to predict these crashes. If the model does not explain part of the variance with the added pandemic data, it can be concluded that it does not play a direct role in price development. To determine the outcome of our research question, we compare the value of our data between a baseline model, linear model and two state-of-the-art (SOTA) models, the random forest regressor (RFR) and auto-regressive integrated moving average (ARIMA) model. Studying the effects of the data between 2019 and 2021 we can not achieve a significant result supporting our hypothesis. Both SOTA models tend to perform better or similar with less features, indicating the data does not add value to the prediction of stock market values.
% Thesis Design
\end{abstract}

%TC:ignore
\keywords{COVID-19, dynamic processes, stock market index, multivariable analysis, time series forecasting}

\maketitle

\section{Introduction}
\label{sec:intro}
% \begin{itemize}
% \item Bevat je onderzoeksvraag (of vragen)
% \item Plaatst je vraag in de bestaande literatuur.
% \end{itemize}
% 
% Je onderzoeksvraag is leidend voor je hele scriptie. Alles wat je doet moet % uiteindelijk terug te voeren zijn op 1 doel: het beantwoorden van die vraag. 
% 
% Typisch zal je het dan ook zo doen:
% 
% Mijn onderzoeksvraag is onderverdeeld in de volgende deelvragen:
% 
% \begin{description}
% \item[RQ1] \ldots We   beantwoorden deze vraag  door het volgende te doen/ antwoord op % de volgende vragen te vinden/ \ldots
% \begin{enumerate}
% \item Vragen op dit niveau kan je echt beantwoorden, en dat doe je in je Evaluatie % sectie~\ref{sec:eva}.
% \end{enumerate}
% \item[RQ2] \ldots
% \item[RQ3] \ldots
% \end{description}
% %
% Je Evaluatie sectie~\ref{sec:eva} bevat evenveel subsecties als je deelvragen hebt. En % in elke sectie beantwoord je dan die deelvraag met behulp van de vragen op het % onderste niveau.
% 
% In je conclusies kan je dan je hoofdvraag gaan beantwoorden op basis van al het eerder % vergaarde bewijs.

At the start and during the COVID-19 pandemic, rising infection rates, deaths and widespread lockdown measures raised questions on how to limit the pandemic's impact on economic development \cite{velavan2020covid}. This thesis looks into how the pandemic affects stock market values, as these are notoriously volatile in their pricing and influenced by a wide range of factors, demonstrated by the WTI value dipping in the negative at the first wave of infections \cite{CORBET2020104978}. 

The relationship between stock markets and economic factors falls under the study of macroeconomics and finds it origins in the business cycle theory, dating back to the early 1800s \cite{de1827nouveaux}. This theory generally states that all economic development follows a cycle or waveform, going through the phases expansion, crisis, recession and recovery. These business cycles can be measured \cite{baxter1999measuring} and are part of the tools that a government can use to mitigate economic downturn. Predicting the development of a business cycle is generally based on economic indicators and economic theory, such as Keynesian theory  \cite{keynes2018general} and real-business cycle (RBC) theory \cite{kydland1982time}. One of the generally accepted economic indicators is the Standard and Poor's (S\&P) 500 as an indicator for the United States (US) economy, stating a relationship between economic development and the top 500 US companies \cite{diblasi_2021}.

One of the complexities of these relationships, between a country and a national representation of the local stock market, is that a stock market is not a local phenomenon. Stocks and other financial products are traded internationally via a stock exchange and these stock indexes, such as the S\&P500, are a representation of the stock exchange itself and not (necessarily) of the country it resides in. As the COVID-19 pandemic's impact and government restrictions varies between countries, it is needed to isolate a region to perform a causal analysis of its impact on a stock market.

As the WTI value dropped, all oil-related stocks dropped as a consequence, such as Royal Dutch Shell, one of the largest companies in the Netherlands. As previous research noted a relationship between pandemic data and economic development, and a relationship between economic development and stock indexes, the question can be posed whether these relationships aid in predictive modelling of stock values on a local level. This research will focus on the Netherlands, with the Amsterdam Exchange index (AEX) as an analog for the S\&P500, the former featuring the top 25 most traded securities on the Amsterdam Exchange. As macroeconomics looks at the large scale development of an economy, the models generally apply to years or decades of data. Day to day stock market development is a lot more volatile than other economic indicators, generally displaying a non-linear patterns. 

This is where machine learning can aid in prediction, as these can capture non-linear behaviour better than traditional linear models \cite{zhang2009stock}. Still, stock market prediction is non-trivial as the efficient market hypothesis (EMH) \cite{peters1996chaos} states that no investor can gain an advantage over others based on historical and current information. Some investors base their decisions on fundamental analysis \cite{lev1993fundamental} to gain an advantage on others. Machine learning has shown to be effective at improving prediction accuracy on stock market securities or crude oil pricing \cite{jain1996, YU20082623, zhang2009stock, thawornwong2004forecasting}, especially using recurrent- (RNN) or artificial neural networks (ANN).

Having established that there is a relationship between stock market values and economic development, the question remains whether the COVID-19 pandemic actually influenced stock market values. Social and economic factors have historically shown to be a important part of stock market crashes, such as with the Tulip Mania \cite{garber1989}, the Wall Street Crash of 1929 and the Great Depression \cite{white1990, kindleberger1986world} and the global financial crisis of 2008 \cite{crotty2009}. An important question during the recent pandemic was whether it would influence the economy and what its impact would be. Historically pandemics have shown their effects \cite{osterholm2017preparing, correia1918pandemics, jorda2020longer}, but the quantitative relationship between the COVID-19 pandemic and economic development is still an object of research \cite{chudik2020economic, baldwin2020economics, fernandes2020economic, deb2020economic}. During the COVID-19 pandemic daily data was registered on its development, such as infection rates, deaths, vaccinations and government restrictions \cite{dong2020interactive, hale2020variation}. As stock market crashes generally are a social phenomenon, this research investigates the relationship between these features and stock market pricing. The following research questions are posed:

% At the start of the pandemic one of the first and unprecedented effects was the dip in the West Texas Intermediate (WTI), a benchmark for crude oil prices. Futures were valued into the negative, meaning distributors of crude oil were essentially paying to deliver oil to accepting customers instead of being paid for the commodity. The pandemic was not the only factor, as Russia and Saudi-Arabia were in a price war for oil, ramping up production and consequently starting a stock market crash due to the big supply and demand gap.

% Along with rising infection rates and wide-spread 'lockdown' measures, the question became how to limit impact of the disease on economic development. In this paper we will look into how the pandemic affects stock market values, as these are notoriously volatile in their pricing and influenced by a wide range of factors, as the WTI demonstrated.

\begin{itemize}
    \item[RQ1] \ldots To what extent does the COVID-19 pandemic data influence stock market values?
        \begin{itemize}
            \item What is the baseline performance for standard statistical models on short term and long term?
            \item How do different statistical and machine learning models compare given the same data?
            \item How do models generalize to different time-periods, such as distinctions between different infection waves?
            \item Which variables in a pandemic affect stock market pricing and to what extent? Variables such as ...
            \begin{itemize}
                \item COVID-19 pandemic data incl. 
                \begin{itemize}
                    \item Infections (provided by the RIVM and JHU)
                    \item Deaths (provided by the RIVM and JHU)
                    \item Vaccinations (provided by the JHU)
                    \item Government restrictions, such as lockdown measures (provided by the OxCGRT)
                \end{itemize}
                \item Environmental factors such as calendar data or weather
            \end{itemize}
        \end{itemize}
    \item[RQ2] \ldots To what extent can machine learning techniques aid in predicting stock market pricing, given a comparison of baseline models and the addition of pandemic data?
\end{itemize}

As the pandemic influenced a wide array of human behaviours, variables are restricted to the ones mentioned above. Other variables, such as mobility, energy consumption and social contacts have shown large differences during the pandemic, but are outside of the scope of this research.

\paragraph{Overview of thesis}
% Hier geef je even kort weer wat in elke sectie staat.

To create context and support of the hypothesis, this paper first discusses related work \ref{sec:rel}. With this context, a proper methodology \ref{sec:meth} is formulated with an explanation on the data collection \ref{sec:data}, model development \ref{sec:models} and how these answer the proposed research questions \ref{sec:eval}. Using the methodology and proper context from related work, results can be analysed \ref{sec:res}, discussed \ref{sec:dis} and be made to construct a conclusion \ref{sec:conc}.


% Related work
% Deze sectie bestaat uit een aantal "blokken", waarin je per blok de relevante literatuur beschrijft. 

% Neem alleen literatuur op die van belang is voor jouw onderzoeksvraag en deelvragen.

% Typisch heb je 1 blok voor je hoofdvraag en per deelvraag \textbf{RQi} een blok. 

% Methodology

% Data verzameling en beschrijving van de data

% Hoe is de data verzameld, en hoe heb jij die data verkregen?

% Wat staat er in de data? Niet alleen maar een technisch verhaal, maar ook inhoudelijk. DE lezer moet een goed idee krijgen over de technische inhoud en wat het betekent.

% Hoe je je vraag gaat beantwoorden.

% Dit is de langste sectie van je scriptie. 

% Als iets erg technisch wordt kan je een deel naar de Appendix verplaatsen. 

% Probeer er een lopend verhaal van te maken.

% Het is heel handig dit ook weer op te delen nav je deelvragen:

% Evaluation

% Met een subsectie voor elke deelvraag.

% In hoeverre is je vraag beantwoord?

% Een mooie graphic/visualisatie is hier heel gewenst.

% Hou het kort maar krachtig.
\section{Related Work}
\label{sec:rel}

% Deze sectie bestaat uit een aantal "blokken", waarin je per blok de relevante literatuur beschrijft. 

% Neem alleen literatuur op die van belang is voor jouw onderzoeksvraag en deelvragen.

% Typisch heb je 1 blok voor je hoofdvraag en per deelvraag \textbf{RQi} een blok. 

As this thesis looks at the influence of COVID-19 pandemic data on stock market values, a couple of sections are dedicated to discuss related work. As the research questions are related to models and data, these are discussed as separate topics.

\subsection{Data}

\subsubsection{Pandemic Data}

As the COVID-19 pandemic began to increase in number of cases, world-wide spread and severity, a large part of the academic world focused their efforts on solving issues related to the pandemic or contributing to the cause in general. One of the most influential parties was the Johns Hopkins University's (JHU) with their Coronavirus Research Center (CRC) and their Center for Systems Science and Engineering (CSSE) \cite{dong2020interactive}. They started one of the first and most successful data collections around the COVID-19 pandemic and provided world-wide figures into the number of infections, deceased and (in a later stage) vaccinations, for varying degrees of granularity. In the early stages of the pandemic, the focus was much more regional \cite{HUANG2020497, wang2020novel, velavan2020covid} and built around statistics such as reproductive numbers, health effects and possible treatments. As the effects of COVID-19 on the community became more apparent and the severity of the pandemic increased, governments became concerned on its effect on society and if their could be economic consequences.

It became clear that lockdown measures were necessary to reduce the number of infections. The exact measures differed from country to country, as the effectiveness of measures were always a matter of debate. Societal measures ranged from curfews mandating people to stay at home and reduce non-essential traveling, mandatory social distancing assuring people refrain from physical contact \cite{lewnard2020scientific} or the wearing of face masks in public spaces to reduce the spread of airborne viral particles \cite{eikenberry2020}. Businesses also faced several restrictions, as the hospitality and tourism industries were forced to close, as they could become hot-spots for disease transmission \cite{gossling2020pandemics}. This in turn also affected the air-travel industry, as people were recommended to stay home and borders were closed off for international travel. And lastly, any other business were often required to provide means for personnel to work remotely from their homes, as offices could also become hot-spots for COVID-19 transmission \cite{dingel2020many}. As these measures varied, especially when each country developed infection peaks at different times, there became a need for COVID-19 related research to have a unified way of charting these measures. The Blavatnik School of Government, part of the University of Oxford, provides a centralized dataset that collects and standardizes the measures taken by countries to reduce the effect of the COVID-19 pandemic \cite{hale2020variation}. They provide categories such as containment and closure policies, economic policies, health system policies, vaccination policies and miscellaneous policies over which they proposed a stringency index that indicates the severity of the measures applied.

\subsubsection{Environmental Data}

As this study revolves around effects on stock market data, environmental factors can not be excluded. Although stock market data are very volatile, their variance can be explained as results from human behaviour. And as humans are affected by a variety of physical and psychological effects, these can be attributed to environmental factors, such as weather \cite{hirshleifer2003good}. For example, without going into the principals behind causality, one might argue that there is a causal relationship between sunny weather and ice-cream sales, as people tend to buy more ice-cream during sunny (summer) days. This is an example of human behaviour being influenced by environmental factors, a phenomenon also influencing decisions of traders on the stock market. Still, human behaviour on the stock market can be erratic as people have to deal with uncertainty \cite{tversky1974judgment} and can overreact to news \cite{de1985does}.

With weather data being location bound, publicly available and easily quantifiable, it serves as an ideal candidate for environmental factors in this study. It does come with some caveats, such as it not always being available historically or not being reliable as some data providers rely on community driven data. Another issue is that weather data is localized, as compared to stock market data, which allows international trade. Still an effect might be noticeable and it serves as a good candidate for correlation and hypothesis testing when compared to pandemic data.

Another large factor in stock market data might be stock related news \cite{tversky1974judgment, veronesi1999stock}. The difficulty of news articles is that they're not easily quantifiable, unless they have been labeled for the specific use case. Unsupervised machine learning methods can be applied to perform sentiment analysis and topic modelling, but this requires large amounts of data and compute resources with low certainty of added value. Textual data also tends to contain a lot of noise, so it requires extensive cleaning before it can be used appropriately. Therefore this research will forego using stock market related news as input data, as it might interfere with the other variables in the research questions.

Another factor in the research questions is the use of calendar data. As stock markets tend to close during weekends and holidays, this data might aid in identifying seasonal patterns. This research will focus on daily and weekly data, as this is the highest frequency data available for stock markets without the use of paid data providers. Pandemic data is also provided on a daily interval, so this coincides fittingly with the other datasets.

% Related work to my thesis
% we discussed the 
% - influence of weather
% - the influence of the pandemic 
% the question remains now
% - What work is related to approach this specific problem
% What does literature say on this?


\subsection{Forecasting models}

As the COVID-19 pandemic is a recent phenomenon, studies modelling economic development after the pandemic have been performed but not yet proven \cite{chudik2020economic, baldwin2020economics, fernandes2020economic}. There have been efforts on modelling historic pandemics \cite{osterholm2017preparing, correia1918pandemics, jorda2020longer}, but most of these studies focus on long-term macroeconomic effects, such as gross domestic product (GDP), unemployment, individual/national income, consumption or inflation, looking at years or decades of development. As economic development itself is a slow process, influence by relatively short-term factors such as the stock market is a non-trivial matter. As modern technology along with the COVID-19 pandemic provides a unprecedented amount of data \cite{dong2020interactive, hale2020variation}, some research has been done on the short-term development and effects of the pandemic \cite{zhao2020preliminary, deb2020economic, carpi2021twitter}.

% To model our data on a daily frequency, we have to look at models that can perform regression tasks. As the data might be linear or non-linear, we have to account for both situations. As a baseline model, we will use a static model that uses the last known value in the response variable as a future prediction. This model will perform consistent regardless of the trend in the data.

A common practice in the development and selection of forecasting models is to perform exploratory data analysis (EDA) and select the model based on an error metric and/or subjective knowledge of the specific domain. Econometric and statistical models generally differ between these disciplines, as the former looks more at domain knowledge and adjusts the model accordingly to reduce error, while the latter relies on statistical methods to reduce model errors. As stock market data is quite volatile, using either of these disciplines can lead to a successful forecasting model, but the theoretical basis differs. As the discipline of data science and machine learning is in essence an evolution of statistics, its methods rely more on error metrics, even though a lot of gains can be made with problem specific model tuning. This study will look solely at the error metrics of models to determine the effectiveness of the model in an automatic manner.

As a baseline comparison, this study has developed a naive model that assumes no trends or variation in the response variable and forecasts the last known value of a time-series. Given the predictor variable will realistically increase or decrease with every time-step $t$, it is assumed it will always show an error, but it will average out over multiple time-series, given normalized data and a relative error metric.
% reference from
% conference
% paper
% book
% github...

% news from conferences/knowledge from books


\subsubsection{Linear model}

The discipline of forecasting goes back to the early 1700s with the prediction of the return of Halley's comet, where orbital mechanics predicted that the comet returned every 76 years. Decades later mathematicians and physicists Legendre, Gauss and Quetelet developed this theory into the basics of linear regression, the standard for most forecasting problems \cite{stigler1986history}. Linear regression (LR) assumes constant variance, independence of errors and linearity of the data. As the data can be transformed to better fit these characteristics, it has been the basic model for most of econometrics \cite{greene2003econometric}. Along with its simplicity, and its computational efficiency, it therefore serves as a good comparison to the baseline model.

\subsubsection{Random Forest Regressor}

Statistical models, although occasionally quite complex, can be explained as a mathematical relationship between explanatory and response variables. Machine learning algorithms, although mathematically motivated, tend to be less clear on their inner workings by being a function of the training data. They tend to work in non-linear situations by combining (relatively) simple mathematical models into ensembles, creating a larger non-linear model, forming the basis of a discipline known as deep learning \cite{lecun2015deep, goodfellow2016deep}. Although most machine learning models require a lot of data to be trained, a selection of models has been made available in a Python library to ease machine learning development \cite{sklearn2012}. One of the models that performs relatively well on small amounts of data is the Random Forest Regressor. A Random Forest is an random ensemble of shallow decision trees, bagged together to create a single model, capable of classification or regression tasks. Properly trained and implemented, they can achieve high accuracy, with the trade-off of lacking interpretability. Where related research implemented neural networks \cite{thawornwong2004forecasting}, these require significant tuning for good performance. Because this study looks at comparing models over a variety of data, this rules out implementing RNNs/ANNs for the sake of generalization.

\subsubsection{ARIMA}

As linear regression can be seen as a basic implementation of regression, the ARIMA model is the most robust regression model that can be used in this context \cite{siaminamini2018forecasting, chatfield2019analysis}. It also allows for automatic fitting which tries to find an optimal fit by stepping through its parameters \footnote{pmdarima: ARIMA estimators for Python: \url{https://alkaline-ml.com/pmdarima}}, although this might lead to over- or underfitting.
\section{Methodology}
\label{sec:meth}

Given the defined research questions and related work to this study, it needs to adhere to the basic steps of forecasting \cite{hyndman2018forecasting, davydenko2021forecast}:

\begin{itemize}
    \item[Data]: Information gathering and EDA
    \item[Models]: Choosing and fitting models
    \item[Evaluation]: Model evaluation methods
\end{itemize}

To determine the extent of influence of COVID-19 pandemic data on stock market pricing, all factors in the questions are isolated and answered separately. The effect of the data can be determined by using comparisons in correlation scores between features or non-parametric tests. This gives a reliable result for one-on-one relationships between explanatory and response variables, but this becomes a lot more complex in multivariate analysis. This study proposes an alternative solution by comparing forecasting performance using a dynamic model framework where feature importance is compared by applying the mean absolute percentage error (MAPE) on a selection of machine learning models, isolating features between comparisons.

$$\mbox{MAPE} = \frac{100}{n}\sum_{t=1}^n  \left|\frac{A_t-F_t}{A_t}\right|$$

To determine the features to be tested, a couple of choices concerning the available data surrounding the COVID-19 pandemic \ref{sec:data} have to be made. As the performance of the data is also dependent on the model trained with it, a collection of models will account for linear- and non-linearity. Statistical dependency can be accounted for by comparing model scores of a LR model with an ARIMA model, as the former assumes statistical dependency \ref{sec:models} and the latter does not. To evaluate performance of the models and data, it is necessary to methodically isolate different parts of the research question given the MAPE metric.

\subsection{Data: Information gathering and EDA}
\label{sec:data}

To fulfill the data requirements, this research is based on a small set of sources:

\begin{itemize}
    \item Calendar data, such as day of the week, weeks, months, years, etc. Its scope will go up to daily data, as that is the finest granularity between all historical data sources.
    \item Weather data, as it has small but noticeable effects on human behaviour. This study will focus its efforts on the Dutch market, where the Royal Netherlands Meteorological Institute (KNMI) is the authority on Dutch weather \footnote{KNMI - Royal Netherlands Meteorological Institute: https://www.knmi.nl/home}, with The Bilt (Utrecht) as the central point of measurements.
    \item Stock market data, collected from the Yahoo Finance API \cite{aroussi_2017}, with the 25 companies representing the Amsterdam Stock Exchange (AEX) index as a representation of the Dutch stock market. As the AEX index is evaluated each quarter, it is assumed as a static composition of companies as of 30th June 2021 \cite{aexindex2021}.
    \item COVID-19 data, provided by
    \begin{itemize}
        \item the Dutch institute of national well-being and environment (RIVM), providing official information on COVID-19 infections and deaths in the Netherlands.
        \item the Johns Hopkins University repository for COVID-19 data \cite{dong2020interactive}, a global dataset providing insight into infections and deaths, but also vaccinations, currently (as of June 2021) not officially provided by the Dutch authorities.
        \item the Oxford COVID-19 Government Response Tracker \cite{hale2020variation}, a global tracker of government responses to the COVID-19 pandemic, including lockdown, social distancing and business measures.
    \end{itemize}
\end{itemize}

% As we are dealing with stock market data, which is notoriously volatile, there are multiple data sources that could aid predictions. Speculation on news publications, especially related to business development of publicly traded stocks, is fairly common in stock exchanges. This data is highly unstructured and requires a lot of manipulation for it to be of value to our models. Therefore we do not include this data, as it complicates model development heavily and could theoretically introduce a lot of noise.

Because of the variety of data sources, it's not likely all data is perfectly clean and distributed for this study. To ensure the models treat all features equally, data has to be preprocessed. During EDA all features were investigated and potential problems were identified with the data before collecting representative results to test our research questions.

\subsubsection{Preprocessing of calendar data}

Calendar data can be generated and as this study models time-series data, it serves as the connection for all other features. It also has its own characteristics which could aid in distinguishing patterns in the data, such as weekdays, boundaries of weeks, months or years and the distinction between workdays and weekends. During EDA it was concluded that these features are counter-productive for the models, as they are discrete variables but produce artifacts at the boundary between periods that do not represent reality. For instance, week 52 is a later than week 1, except at the transition of two years. This could be resolved by counting from an arbitrary starting point in the data and only counting up, but generally this just produces more features that are confusing to the models. The days of the week are different, as these are more categorical in nature rather than discrete variables. For instance when every Monday is counted incrementally, it would represent the number of weeks rather than every Monday. Therefore its better to one-hot encode weekdays.

% TODO: Check whether incremental discrete variables for day/week/month/year do work
% Not sure is this makes sense

\subsubsection{Preprocessing of weather data}

Weather data consists of a few basic variables: air humidity and -pressure, wind speed, cloud coverage and conversely hours of sunshine, precipitation and temperature. All these features have their respective minimum and maximum values with the time-slot where that value was measured and an average or total sum depending on the type of variable. As the granularity of the data does not allow for hourly intervals and extreme values might badly influence model training, this study restricts the weather features to averages and totals of the main features mentioned.

\subsubsection{Preprocessing of stock market data}

Stock market data consists of two basic variables: pricing and volume. Pricing is generally represented by a opening- and closing price, a high and a low value per day. In stock trading average values tend to be less informative, as it is the extreme differences in value between certain periods that are of value for stock trading. There are numerous metrics that exist within economics and financial theory that claim to be good indicators of the actual value of an asset, but only open, close, high and low values along with volume are agreed upon as being representative indicators. As all other constructions of these indicators are open to speculation, this study treats these values individually as predictor values.

A different challenge with stock market data is that it is not always available, as a stock market is not always open. Weekends, holidays and other periods might not contain data, so with preprocessing these are indicated as a variable in the dataset. As zero-filling is not representative of these variables, it is assumed that the value of prices or volume remain static during these periods. This has its dangers as the world does not 'switch off' during these periods and speculation on share prices will still occur, without it being reflected in actual trading prices. Although recording the closing days of the stock market might help models detect and predict these periods, it will inevitably introduce noise to the predictions and it might be necessary to move toward a weekly prediction interval rather than a daily interval.

\subsubsection{Preprocessing of pandemic data}

COVID-19 data features four variables: infected, deceased and vaccinated people, along with government measures. All these can either be viewed as daily recorded values or cumulative values, with the exception infected people, as the duration of COVID-19 infection varies, generally showing a two week average duration. Data is provided by two sources, the RIVM and the JHU, where the former can be viewed as reliable and the latter can be changeable, as the data is crowd-sourced. Both suffer from the flaw that the recording of a value might differ from the actual value at that time. Infected people might not always know they are infected and their infection status is only recorded if they decide to be officially tested. A similar phenomenon occurs with the number of deceased people, as the cause of death may not always be determined (to a single cause), people are not required to notify the government and the deceased are recorded by this notification date and not time of death. Vaccination data features these same inconsistencies, but also has the deficit of being crowd-sourced from various sources at the start of recording, as the RIVM did not officially record these statistics at the start of this study. Therefore this study will rely on the vetting mechanism that open-source data applies, accepting any errors this may introduce.

Some gaps in the data may occur and with cumulative values a 'steady state' is assumed, meaning no change is expected in the state unless recorded. For daily recorded statistics this means data is zero-filled if no value is known. This differs from the approach with stock market data, as vaccinations could be halted due to political or medical re-evaluations.\footnote{The AstraZeneca and Janssen vaccines were both found to cause embolic/thrombotic events in patients and were suspended until safety was reassessed: \\ \url{https://www.cdc.gov/vaccines/acip/meetings/downloads/slides-2021-01/02-COVID-Villafana.pdf}\\  \url{https://www.fda.gov/news-events/press-announcements/joint-cdc-and-fda-statement-johnson-johnson-covid-19-vaccine}}
Although it can be assumed infections and deaths maintain relatively stable figures, this study is aware that this data is curated and historically corrected, removing the necessity for gap-filling.

All data is prepared in a similar fashion before model training and evaluation. Because it can not be assumed that features are homogeneous, min-max-normalization is applied on the data between all transformations so that each feature is scaled identically without modifying their distribution. As for transforming the distribution of non-normally distributed data, it was concluded from EDA that applying a log-transformation, a second re-scaling along with a Yeo-Johnson transformation \cite{yeo2000new}, yields the best results for data distribution, while maintaining distribution characteristics for normally distributed data. Finally the data is re-scaled once more before model training so each feature will be weighted similarly. As the response variable this study chose to look at opening prices of stocks, as a single variable allows for easier error metrics. From EDA it was concluded that there is little difference in either opening, closing, high or low pricing values.

\subsection{Feature selection}

As the principal of 'simple is better' generally applies to time-series forecasting, it can not be assumed that all features combined as the full training set is best. To remove these assumptions, recursive feature elimination is implemented during model training based on the mean squared error (MSE) score of a prediction with five fold cross validation. However, this only applies for the LR and RFR models as the used ARIMA model implementation does not allow for automatic feature selection.\footnote{pmdarima: ARIMA estimators for Python \url{https://alkaline-ml.com/pmdarima/index.html}} For the ARIMA model a Fourier Featurizer is implemented to fit seasonal as well as non-seasonal features effectively. And lastly our baseline model does not necessitate any feature selection as it disregards all input features other than the last value of the response variable, which results as a constant for each training cycle.

\subsection{Models: Choosing and fitting models}
\label{sec:models}

To accommodate a valid conclusion to the research questions, a variety of models must be considered  and compared in their individual performances to the larger research questions. Only then can this study conclude if it is the model that accounted for a gain in performance.

\subsubsection{Baseline model}

For our baseline model, it is necessary to ensure the performance is consistent irrespective of the input data. Effectively this means the baseline model should predict a constant value from the last known value in a time-series. Assuming the error of the predicted values is normally distributed, the MAPE metric should be relatively consistent over multiple predictions and inputs.

$$\hat{Y}_{T+h|T} = Y_T$$

For every time-step $t$ and feature $i$, we define the predictor $y^{t+1}$ as the last time-step $t$ of input value $x$.

\subsubsection{Linear regression}

As the standard for most prediction models, a model based on multiple linear regression (LR) is implemented. As the model expects constant variance, independence of errors and linearity of the data, it will only perform adequately if these conditions are met. With specific transformations applied to the data to improve model fitting, it serves as a good comparison to the other models as its value has been proven in many fields and disciplines. 

\subsubsection{ARIMA}

As a more advanced regression model, ARIMA has the potential to fit the data quite well. As an auto-regressive (AR) model, it does not assume independent data and serves as a good comparison to LR on this feature of the model. The moving average (MA) accounts for the errors in the AR part of the model, improving the model fitting. Still, this ARMA model does not work on non-stationary data, which is where the (I) integrated part of the model comes in. This accounts for the difference in time-steps ($Y_t+1 - Y_t$) in the data and transforms the series so it does become stationary for the ARMA part to work. The key takeaway being that the three parameters $p$, $d$ and $q$, respectively representing AR, I and MA, need to be tuned for a correct model fitting. This can be performed automatically by performing differencing tests for the $p$ parameter and iterating step-wise through the $d$ and $q$ parameters. This could lead to over-fitting the model, but with multiple samples of the data this study should get representative model performance.

\subsubsection{Random Forest Regressor}

As a state-of-the-art (SOTA) model a random forest regressor (RFR) is implemented. It can fit non-linear data, but is a 'black-box' model, so compared to LR and ARIMA its inner workings are not fully intuitive. But where it lacks in explainability it gains in performance. Related literature also points out that RNN/LSTM networks can perform well on stock market time-series data \cite{YU20082623}, but these networks require quite large amounts of data compared to an RFR model. Therefore it is the best machine learning model we could implement on our data.

\subsection{Evaluation: Model evaluation methods}
\label{sec:eval}

To evaluate model performance it is compared between different segments of the research questions using the MAPE metric. In these questions the following components can be isolated:

\begin{itemize}
    \item Model performance
    \item The time period of the sample
    \item Feature performance
\end{itemize}

Using these cross-sections of the research questions a statistical hypothesis test can be performed on the MAPE metric to draw a valid conclusion. When a model will not significantly outperform the baseline model, this study has to acknowledge that the research question does not have a conclusive answer given the data and models.
\section{Results}
\label{sec:res}

% Template tips:
% \begin{itemize}
%     \item Met een subsectie voor elke deelvraag.
%     \item In hoeverre is je vraag beantwoord?
%     \item Een mooie graphic/visualisatie is hier heel gewenst.
%     \item Hou het kort maar krachtig.
% \end{itemize}

% include somehwere
% \cite{li2020choose}

% Include reference to github repository with basic instructions on how to load data, models, results and evaluation for reproducability

% For visuals try to look at color maps or heatmaps

Model training is performed for all 25 companies in the AEX index, with five randomly sampled time periods for each of the models. This ensures there is enough data to answer the research questions. Although all parameters are kept consistent between model training, some errors do inevitably creep into the results necessitating result selection. MAPE scores equal or above 1.0, i.e. a +100\% error rate, are discarded, as these are obvious errors where the models did not perform as expected. With identical model training cycles, random samples seeded with identical values and identical training data, identical results were expected, but the reality is that results have to be manually selected to ensure quality. To assure each model is represented equally, 50 results were sampled from each model and plot our results.

\subsection{RQ1: The influence of pandemic features on stock pricing}

To fully answer the extent of this research question, first the performance of our models must be assessed in comparison to the baseline. During EDA and the results collection a sample of these results was taken from the training cycles and pair-plotted for quick comparison (see Figure \ref{fig:samplefits}).

% \begin{figure}[H]
%     \includegraphics[width=.4\paperwidth]{images/plots/sample_fits_asml.png}
%     \caption{Side-by-side model fittings and resulting daily predictions for ASML}
%     \label{fig:samplefits}
% \end{figure}

During EDA this study experimented with different sized ratios of training and testing data and determined a ten to one ratio of training vs. testing data should be sufficient. The target prediction length was seven days of values for the daily interval and three weeks of values for weekly interval. From the pair-plot model fittings \ref{fig:samplefits} some prediction errors can clearly be seen, but these are expected given the variance in the training data. It can also clearly be seen that the baseline model (labeled as "dummy") functions as expected, ignoring all input features except the last known value of the response variable.

\subsubsection{Model performance}

Given access to all features while implementing automatic feature selection, model performance on the MAPE metric is compared for daily and weekly data (see Figure \ref{fig:modelcompare}).

\begin{figure}[H]
    \includegraphics[width=.4\paperwidth]{images/plots/base_model_comparison_daily_weekly.png}
    \caption{Comparison of models with daily data vs. weekly data, with attained MAPE scores}
    \label{fig:modelcompare}
\end{figure}

Weekly data generally shows higher error scores, especially for the ARIMA model. An unusual result is that none of the models show any noteworthy performance differences when compared to the baseline model. The baseline should average out to a small error score, but the expectation was that with sufficient training data and feature selection, the other models should've outperformed the baseline. In Figure \ref{fig:samplefits} a possible cause can be seen, as some results show extreme errors in predictions.

\subsubsection{Time period}
As a difference between daily and weekly data was established for predictions, the results can also distinguish the used time period for training and testing.

\begin{figure}[H]
    \includegraphics[width=.4\paperwidth]{report/images/plots/daily_2019_vs_2020_vs_2021_max_05.png}
    \caption{Models trained on daily data, split by year, capped at MAPE score <= 0.5}
    \label{fig:yearplot}
\end{figure}

As the COVID-19 pandemic started at the end of 2019, it can be seen that the error scores increase in 2020 and consequently decrease in 2021. This could support the hypothesis that the pandemic caused an upset in 2020 and was restoring in 2021. An indication that this is not just an artifact of the pandemic data, there is little difference to be seen in the baseline model between 2019 and 2021, even though the pandemic is still widespread in the latter year. The same behaviour can also be seen on a weekly basis, indicating that this also generalizes over longer prediction periods.

\subsubsection{Feature performance}
To isolate the impact of data features on our error metric, the same predictions were ran given different selections of features. The results above were given full access to the data, but the following variables were isolated:

\begin{itemize}
    \item Weather features
    \item COVID-19 infections
    \item COVID-19 deaths
    \item Partial and full COVID-19 vaccinations
    \item Stringency of COVID-19 regulations \cite{hale2020variation}
    \item No additional features
\end{itemize}

Looking at a comparison of the MAPE scores between weather data only and the full dataset, there is no notable difference in performance (see Figure \ref{fig:weathervfull}). This could indicate that either the weather features are always selected during features selection, or none of the features are of any value to the models.

\begin{figure}[H]
    \includegraphics[width=.4\paperwidth]{report/images/plots/daily_full_covid_data_vs_weather_data_only_max_05.png}
    \caption{Comparison of model scores between the full dataset and weather data only}
    \label{fig:weathervfull}
\end{figure}

The latter could be true given none of the models show a notable performance increase from the baseline, but the feature of COVID-19 infections shows that pandemic data does have some influence (see Figure \ref{fig:infectvfull}), at least for the ARIMA and RFR models. The same can be said for COVID-19 deaths, vaccination data and the stringency index, where the ARIMA and RFR model (almost) outperform the baseline model. Although the differences are subtle, they are noticeable, especially compared to LR which shows consistent high error scores compared to the baseline.

\begin{figure}[H]
    \includegraphics[width=.4\paperwidth]{report/images/plots/daily_full_covid_data_vs_infected_data_only_max_05.png}
    \caption{Comparison of model scores between the full dataset and COVID-19 infection data only}
    \label{fig:infectvfull}
\end{figure}

Weather features are the richest of all data and COVID-19 features can be seen as individual features. The only test left is one where the models are given no additional input features and only the response variable defines the output of the models. As a consequence all models increase in performance compared to the baseline as the number of data features decrease, posing the question whether any additional features is helpful at all (see Figure \ref{fig:indexvfull}. 

\begin{figure}[H]
    \includegraphics[width=.4\paperwidth]{report/images/plots/daily_full_covid_data_vs_index_data_only_max_05.png}
    \caption{Comparison of model scores between the full dataset and no additional features}
    \label{fig:indexvfull}
\end{figure}

\subsection{RQ2: Performance of machine learning techniques in stock market prediction}
As the RFR model is considered as part of the machine learning techniques, it should be compared to the baseline and evaluated on its relative performance. The differences in performance are minute and necessitate an insight into the significance tests to give a reliable measure of difference in performance. For the sake of simplicity, this study has only done this comparison for the daily data, as the weekly data shows unreliable performance. Given a confidence level of 95\% ($\alpha = 0.05$), a T-test on the MAPE scores for the full dataset and the dataset containing no additional features results in the following table \ref{tbl:modelscores}.

\begin{table}[H]
\begin{tabular}{lllll|}
\cline{4-5}
\multicolumn{3}{l|}{}                                                                                                                        & \multicolumn{2}{l|}{\textbf{Baseline comparison}} \\ \hline
\multicolumn{1}{|l|}{\textbf{Featureset}}                    & \multicolumn{1}{l|}{\textbf{Model}} & \multicolumn{1}{l|}{\textbf{Avg. MAPE}} & \multicolumn{1}{l|}{\textbf{t-stat}}  & \textbf{p} \\ \hline
\multicolumn{1}{|l|}{\multirow{2}{*}{\textbf{All}}} & \multicolumn{1}{|l|}{\textbf{ARIMA}}                      & 0.0302                                  & -0.3735                               & 0.7096     \\ \cline{2-5} 
\multicolumn{1}{|l|}{}                                       & \multicolumn{1}{|l|}{\textbf{RFR}}                        & 0.0234                                  & 0.6711                                & 0.5037     \\ \hline
\multicolumn{1}{|l|}{\multirow{2}{*}{\textbf{None}}}  & \multicolumn{1}{|l|}{\textbf{ARIMA}}                      & 0.0315                                  & -1.7365                               & 0.0856     \\ \cline{2-5} 
\multicolumn{1}{|l|}{}                                       & \multicolumn{1}{|l|}{\textbf{RFR}}                        & 0.0242                                  & -0.5310                               & 0.5966     \\ \hline
\end{tabular}
\caption{Model MAPE score comparison to baseline, for all dataset features and no additional features}
\label{tbl:modelscores}
\end{table}

With these results it can be said that there is no significant performance difference in using machine learning techniques such as RFR for stock market prediction. The same can be said for the ARIMA model, with the notion that ARIMA does seem to show better performance with reduced features. As each row in table \ref{tbl:modelscores} refers to an individual T-test, the dimensions can also be flipped to test whether the features in the data influence the models in table \ref{tbl:datascores}.

\begin{table}[H]
\begin{tabular}{l|llll|}
\cline{2-5}
                                        & \multicolumn{1}{l|}{\textbf{All features}} & \multicolumn{1}{l|}{\textbf{No features}} & \multicolumn{2}{l|}{\textbf{Comparison}}          \\ \hline
\multicolumn{1}{|l|}{\textbf{Model}}    & \multicolumn{1}{l|}{\textbf{Avg. MAPE}}    & \multicolumn{1}{l|}{\textbf{Avg. MAPE}}   & \multicolumn{1}{l|}{\textbf{t-stat}} & \textbf{p} \\ \hline
\multicolumn{1}{|l|}{\textbf{ARIMA}}    & 0.0341                                     & 0.0434                                    & -0.8300                              & 0.4114     \\ \hline
\multicolumn{1}{|l|}{\textbf{RFR}}      & 0.0300                                     & 0.0408                                    & -0.7865                              & 0.4380     \\ \hline
\multicolumn{1}{|l|}{\textbf{Baseline}} & 0.0475                                     & 0.253                                     & 1.3612                               & 0.1824     \\ \hline
\end{tabular}
\caption{Data feature MAPE score comparison for all the ARIMA, RFR and baseline models}
\label{tbl:datascores}
\end{table}

This research shows there is no significant difference in the mean MAPE error scores between all or no additional features in the data, for none of the models.
\input{conclusions}

% your refs

\printbibliography
% \bibliographystyle{plain}
% \bibliography{MyThesis}

% \appendix

% Remember to include annex:
% - EDA and prework
%   - Refer to annex when including partial results or when motivating choices in methodology

\clearpage

\onecolumn

\appendix

\section{Appendix}

\subsection{Plots}

\begin{figure}[H]
  \includegraphics[width=\textwidth]{images/plots/sample_fits_asml.png}
  \caption{Side-by-side model fittings and resulting daily predictions for ASML}
  \label{fig:samplefits}
\end{figure}

\pagebreak

\subsubsection{Daily plots}

\twocolumn

\begin{figure}[H]
    \includegraphics[width=.4\textwidth]{report/images/plots/daily_2019_vs_2020_vs_2021_max_05.png}
    \caption{Comparison of models for daily data over three years}
\end{figure}

\begin{figure}[H]
    \includegraphics[width=.4\textwidth]{report/images/plots/daily_full_covid_data_vs_deceased_data_only_max_05.png}
    \caption{Comparison for daily data of deceased vs. all COVID-19 data}
\end{figure}

\begin{figure}[H]
    \includegraphics[width=.4\textwidth]{report/images/plots/daily_full_covid_data_vs_index_data_only_max_05.png}
    \caption{Comparison for daily data of only index data vs. all COVID-19 data}
\end{figure}

\begin{figure}[H]
    \includegraphics[width=.4\textwidth]{report/images/plots/daily_full_covid_data_vs_infected_data_only_max_05.png}
    \caption{Comparison for daily data of infected vs. all COVID-19 data}
\end{figure}

\begin{figure}[H]
    \includegraphics[width=.4\textwidth]{report/images/plots/daily_full_covid_data_vs_stringency_data_only_max_05.png}
    \caption{Comparison for daily data of stringency index vs. all COVID-19 data}
\end{figure}

\begin{figure}[H]
    \includegraphics[width=.4\textwidth]{report/images/plots/daily_full_covid_data_vs_vaccinated_data_only_max_05.png}
    \caption{Comparison for daily data of vaccinated vs. all COVID-19 data}
\end{figure}

\begin{figure}[H]
    \includegraphics[width=.4\textwidth]{report/images/plots/daily_full_covid_data_vs_weather_data_only_max_05.png}
    \caption{Comparison for daily data of weather vs. all COVID-19 data}
\end{figure}

\pagebreak

\subsubsection{Weekly plots}

\begin{figure}[H]
    \includegraphics[width=.4\textwidth]{report/images/plots/weekly_2019_vs_2020_vs_2021_max_05.png}
    \caption{Comparison of models for weekly data over three years}
\end{figure}

\begin{figure}[H]
    \includegraphics[width=.4\textwidth]{report/images/plots/weekly_full_covid_data_vs_deceased_data_only_max_05.png}
    \caption{Comparison for weekly data of deceased vs. all COVID-19 data}
\end{figure}

\begin{figure}[H]
    \includegraphics[width=.4\textwidth]{report/images/plots/weekly_full_covid_data_vs_index_data_only_max_05.png}
    \caption{Comparison for weekly data of only index data vs. all COVID-19 data}
\end{figure}

\begin{figure}[H]
    \includegraphics[width=.4\textwidth]{report/images/plots/weekly_full_covid_data_vs_infected_data_only_max_05.png}
    \caption{Comparison for weekly data of infected vs. all COVID-19 data}
\end{figure}

\begin{figure}[H]
    \includegraphics[width=.4\textwidth]{report/images/plots/weekly_full_covid_data_vs_stringency_data_only_max_05.png}
    \caption{Comparison for weekly data of stringency index vs. all COVID-19 data}
\end{figure}

\begin{figure}[H]
    \includegraphics[width=.4\textwidth]{report/images/plots/weekly_full_covid_data_vs_vaccinated_data_only_max_05.png}
    \caption{Comparison for weekly data of vaccinated vs. all COVID-19 data}
\end{figure}

\begin{figure}[H]
    \includegraphics[width=.4\textwidth]{report/images/plots/weekly_full_covid_data_vs_weather_data_only_max_05.png}
    \caption{Comparison for weekly data of weather vs. all COVID-19 data}
\end{figure}

\section{Slides}

To be added.

% Example
% \includepdf[nup=2x3 , pages=-]{sargent-lecture_slides.pdf}
 
\end{document}
